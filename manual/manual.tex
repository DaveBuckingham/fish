\documentclass[11pt,letterpaper,article,oneside]{memoir}
\usepackage[utf8]{inputenc}
\usepackage[T1]{fontenc}
\usepackage{microtype}
\usepackage[dvips]{graphicx}
\usepackage{xcolor}
\usepackage{times}

\usepackage{booktabs}

\usepackage{enumitem}
\setlist[description]{style=nextline}
\setlist[itemize]{nosep}

\usepackage[
breaklinks=true,colorlinks=true,
linkcolor=blue,urlcolor=blue,citecolor=blue,% PDF VIEW
%linkcolor=black,urlcolor=black,citecolor=black,% PRINT
bookmarks=true,bookmarksopenlevel=2]{hyperref}

\usepackage{geometry}
% PDF VIEW
% \geometry{total={210mm,297mm},
% left=25mm,right=25mm,%
% bindingoffset=0mm, top=25mm,bottom=25mm}
% PRINT
\geometry{total={210mm,297mm},
left=20mm,right=20mm,
bindingoffset=10mm, top=25mm,bottom=25mm}

\OnehalfSpacing

%%% STYLE OF SECTIONS, SUBSECTIONS, AND SUBSUBSECTIONS
\setsecheadstyle{\large\bfseries\raggedright}
\setsubsecheadstyle{\bfseries\raggedright}


%%% STYLE OF PAGES NUMBERING
\pagestyle{plain}
\makepagestyle{plain}
\makeevenfoot{plain}{\thepage}{}{}
\makeoddfoot{plain}{}{}{\thepage}
\makeevenhead{plain}{}{}{}
\makeoddhead{plain}{}{}{}

\maxsecnumdepth{section}
\maxtocdepth{section}



\newcommand{\name}{PROGRAM\_NAME}
\newcommand{\programVersion}{0.1}
\newcommand{\manualVersion}{0.1}
\newcommand{\email}{eric.tytell@tufts.edu}

\newcommand{\csv}{\texttt{.csv}}
\newcommand{\hdf}{\texttt{.hdf5}}


\renewcommand{\arraystretch}{1.2}

\setlength{\parindent}{0em}
\nonzeroparskip





\begin{document}

\thispagestyle{empty}

{%%%
\centering
\Large

\vspace*{\fill}

{\huge
\name{} \programVersion{}
}

{\LARGE
%User manual version \manualVersion{} \\
User manual \\
\vspace{2.5cm}
First Author \\
Second Author \\
Another Author \\
}
\vspace*{\fill}

}

\cleardoublepage

\tableofcontents*

\clearpage



%%%%%%%%%%%%%%%%%%%%%%%%%%%%%%%%%%%%%%%%%%%%%%
%                 LICENSE                    %
%%%%%%%%%%%%%%%%%%%%%%%%%%%%%%%%%%%%%%%%%%%%%%

\chapter{Copyright and License}

\name{}: a tool for collecting IMU measurements.
Copyright (C) 2017 Author(s)????????/

This program is free software: you can redistribute it and/or modify
it under the terms of the GNU General Public License as published by
the Free Software Foundation, either version 3 of the License, or
(at your option) any later version.

This program is distributed in the hope that it will be useful,
but WITHOUT ANY WARRANTY; without even the implied warranty of
MERCHANTABILITY or FITNESS FOR A PARTICULAR PURPOSE.  See the
GNU General Public License for more details.

You should have received a copy of the GNU General Public License
along with this program.  If not, see <http://www.gnu.org/licenses/>.



%%%%%%%%%%%%%%%%%%%%%%%%%%%%%%%%%%%%%%%%%%%%%%
%               INTRODUCTION                 %
%%%%%%%%%%%%%%%%%%%%%%%%%%%%%%%%%%%%%%%%%%%%%%

\chapter{Introduction}

\name{} is an application for recording and processing data from Inertial Measurement
Units (IMUs). It has two parts. Microcontroller code running on an Arduino
board collects data from the IMUs and transmits it to a PC. A Python program
running on the PC receives data from the Arduino and provides users with a
graphical interface. This interface allows users to interact with the Arduino
and to view and manipulate IMU data.



%%%%%%%%%%%%%%%%%%%%%%%%%%%%%%%%%%%%%%%%%%%%%%
%             SECTION: HARDWARE              %
%%%%%%%%%%%%%%%%%%%%%%%%%%%%%%%%%%%%%%%%%%%%%%

\chapter{Hardware}

\name{} interconnects three pieces of computational hardware: a PC, an Arduino,
between 1 and 3 IMUs.

\section{PC}
\name{} has been tested on PCs running Linux Mint 18 Sarah, OS X El Capitan
10.11, and Windows 10. It is expected to work with other versions of these
operating systems and with other Linux distributions.

\section{Arduino}
\name{} has been tested with an Arduino UNO. It may work on other
models having a 16MHz clock speed and sufficient storage.

\section{IMUs}
The mpu9250 by InvenSense is a nine-axis (gyroscope, accelerometer, compass)
motion tracking device.  For testing purposes, or if the added size and weight
are acceptable for an application, an mpu9250 mounted on a circuit-board can be
used with \name{}.

For applications requiring minimum package size and weight, it is possible to
wire the mpu9250 directly to the Arduino. A separate manual, \textbf{name},
documents the procedure we have used to prepare the mpu9250 for use with
\name{}.

\section{Wiring the Arduino}
\label{sec:wiring}
Table \ref{tab:wiring} summarizes the process of connecting IMUs to the
Arduino. It may be necessary to use a breadboard, especially if multiple IMUs
are used. Figure \ref{fig:wiring} shows an Arduino wired to a single IMU.

The Arduino communicates with the IMUs using the Serial Peripheral Interface bus
(SPI) protocol. Each IMU needs a separate \emph{chip select} line, but all IMUs
share the other lines. Pins 8, 9, and 10 are used for \emph{chip select} lines
for up to three IMUs. Pin 11 carries data traveling from the Arduino to the IMUs,
i.e. Master-Out, Slave-In (MOSI). Pin 12 carries data traveling from the IMUs to
the Arduino, i.e. Master-In, Slave Out (MISO). Pin 13 carries a clock signal
which regulates timing of the communication protocol.  Each IMU should be
connected to 3.3V power (available on the Arduino) and to ground.

If using an optional trigger, connect it to Pin 4 and to Ground.

Connect the Arduino to the PC with a USB cable.  To ensure adequate power,
especially if using multiple IMUs, it is recommended to power the Arduino with
an external power source instead of relying on the USB port.

\begin{table}
\centering
\begin{tabular}{@{}*4l@{}}
\toprule
description & label & color & pin \\
\midrule 
IMU 1 chip select & NCS / CS & white & 8 \\
IMU 2 chip select & NCS / CS & white & 9 \\
IMU 3 chip select & NCS / CS & white & 10 \\
data from Arduino to IMU & MOSI / SDA & green & 11 \\
data from IMU to Arduino & MISO / SDO & blue & 12 \\
clock & SCL / CLK & yellow & 13 \\
power &  & red & 3.3V \\
ground &  & black & GND \\
trigger &  &  & 4 \\
\bottomrule
\end{tabular}
\label{tab:wiring}
\end{table}

\begin{figure}[]
    \begin{center}
        \includegraphics[height=3in]{wiring2}
    \end{center}
    \caption{Wiring a single IMU to the Arduino}
\end{figure}



%%%%%%%%%%%%%%%%%%%%%%%%%%%%%%%%%%%%%%%%%%%%%%
%          SECTION: INSTALLATION             %
%%%%%%%%%%%%%%%%%%%%%%%%%%%%%%%%%%%%%%%%%%%%%%

\chapter{Installation}

\section{Arduino IDE}
While there are many tools for installing program code onto the Arduino, we have
tested \name{} using the Arduino IDE.

The software can be downloaded from:
\url{https://www.arduino.cc/en/Main/Software}

If your operating system has a package management system, it might be able
to install the Arduino IDE. For example, on a Debian-based system you can use
\texttt{apt}:
\begin{verbatim}
# apt install Arduino-core
\end{verbatim}

\section{Install \name{} on the Arduino}
\label{sec:installarduinocode}

Install the file \verb|am_tx/am_tx.ino| on the Arduino.
This can be accomplished using the Arduino IDE graphical user interface.

Alternatively, using Arduino IDE version 1.5.0 or later,
the Arduino can be programmed directly from the command line:

\begin{verbatim}
# arduino --upload am_tx/am_tx.ino
\end{verbatim}

For more information about the Arduino command line interface, see:
\url{https://github.com/arduino/Arduino/blob/master/build/shared/manpage.adoc}

\section{PC}
\name{} requires Python3.

We recommend using pip to install \name{}:
\url{https://pypi.python.org/pypi/pip}

\begin{verbatim}
# pip install --upgrade pip
# pip install \name
\end{verbatim}

Pip will automatically install any of the following dependencies if needed:
\begin{itemize}
\item h5py
\item numpy
\item pyqtgraph
\item pyserial
\item pyqt5
\end{itemize}

Once \name{} has been installed, it can be started from the command line:\\
\texttt{
\# \name{}
}

\begin{figure}[]
    \begin{center}
        \includegraphics[height=3in]{screenshot_panel}
    \end{center}
    \caption{Control panel} 
\end{figure}


\chapter{The data buffer}
\name{} uses a single data buffer to hold IMU data. When data is recorded from
the IMUs, the data buffer is first erased. Then each sample is added to the
buffer as it is recorded. The buffer can be saved to a file, or a file can be
loaded into the buffer, erasing any previous contents. Data processing
operations can be performed on the data buffer, altering it irreversibly.
Therefore, any time the data buffer contains valuable data it is recommended to
save it to file before collecting new data, loading another file, or executing
data processing operations.

The ``data buffer length (\# samples)'' slider adjusts the size of the data
buffer. When each new sample is received, it is added to the data buffer. If the
buffer was already full (the number of samples in the buffer was equal to the
size of the buffer), the oldest sample in the buffer is deleted. If the size of
the buffer is adjust to be smaller than the number of samples in the buffer, the
oldest samples in the buffer are deleted until the number of samples in the
buffer is equal to the buffer size.


\chapter{Collecting data}

To begin collecting data, press the \emph{record} button.  The \emph{record} button
will change into a \emph{stop} button.  The PC will establish communication with to the
Arduino and instruct it to begin collecting data from the IMUs at 200Hz.

Any samples received from the Arduino are stored in the data buffer. As
samples are recorded in the data buffer, they become visible in the data
visualization plots. For each IMU, one row of three plots is displayed. The
left-most plot shows accelerometer data, the center plot shows gyroscope data,
and the right-most plot shows magnetometer data. For each plot, red, green, and
blue lines show x, y, and z axis measurements, respectively.

The \emph{stop} button (or the trigger, as described in Section \ref{sec:trigger}),
causes the PC first to instruct the Arduino to stop collecting data and then to
halt communication with the Arduino. The \emph{stop} button will change back into the
\emph{record} button.

\begin{figure}[]
    \begin{center}
        \includegraphics[width=\textwidth]{screenshot_plots}
    \end{center}
    \caption{Collecting data from a single IMU} 
\end{figure}



\section{Trigger}
\label{sec:trigger}

Optionally, a trigger attached to the Arduino (Section \ref{sec:wiring}) can be
used to stop recording.  If the \emph{use trigger} checkbox is not checked, the
trigger is ignored. Otherwise, if an active trigger is detected, recording
will stop, i.e., the same effect as pressing the \emph{stop} button during
recording. If you attempt to begin recording while the \emph{use trigger} checkbox
is checked and the trigger is active, the recording end as soon as it starts
with zero data stored.

If the \emph{invert trigger} checkbox is not checked, the trigger is considered
``active'' when the associated Arduino pin is set high. If the \emph{invert
trigger} checkbox is checked, the trigger is considered active when the
associated Arduino pin is set low.

If there is nothing setting the current on the Arduino pin associated with the
trigger (i.e. if no trigger is connected), then the value of the pin is
undefined.  Thus, to ensure reliable behavior, the \emph{use trigger} checkbox
should not be checked unless there is a trigger connected to the Arduino.




\chapter{Saving and loading}
\label{sec:savingloading}

\section{File formats}

\label{sec:fileformats}
The \csv{} file format stores data in plain text, using newlines to delimit
samples and comas to delimit measurements within each sample.  \name{} uses the
CSV module in the Python Standard Library with default formatting parameters:
\url{https://docs.python.org/3/library/csv.html}

The \hdf{} format is designed to store large amounts of data. It is generally more
compact than \csv{}. \name{} uses the h5py package to read and write \hdf{} files:
\url{http://www.h5py.org/}


\section{Saving}

When you press the \emph{save} button, a dialog will appear allowing you to
select a filesystem location, a filename, and a filetype (either \csv{} or
\hdf{}).  The the data buffer is saved to file according to the selections made
in the dialog. The \emph{save} button is disabled when the data buffer is empty
and while data is being recorded.

\section{Loading}

When you press the \emph{load} button, a dialog will appear allowing you to
select a filetype (either \csv{} or \hdf{}) and a file to load. The file will be
loaded into the data buffer, overwriting any data stored there.
The \emph{load} button is disabled while data is being recorded.


\chapter{Processing data}

\begin{figure}[]
    \begin{center}
        \includegraphics[height=3in]{screenshot_process}
    \end{center}
    \caption{Data processing control panel} 
\end{figure}

The \emph{process} button opens the data processing control panel.

\section{Process mode}
All data transformation algorithms are applied to the data buffer. If the \emph{use
current data} radio button is selected, the current contents of the data buffer
are used. This option is not available when the data buffer is empty.

Alternatively, if the \emph{Batch process} radio button is selected,
the contents of the data buffer are ignored. Instead, batch processing iterates
over a list of one or more input files, loading each file into the data buffer,
applying a transformation algorithm, and saving the result to a new output file
before proceeding to the next input file in the list. The \emph{Select files}
button launches a dialog for selecting input files for batch processing.
The batch processing output filetype can be set to \hdf{}, \csv{}, or can vary to
match the filetype of each input file. The output suffix string is appended to
the filename of each input file (before the file extension) to compose the
output filename. If the output suffix string is empty and the output
filetype matches the input filetype, then the input file is overwritten.

With either processing mode, the contents of the data buffer are overwritten by
any data processing.

\section{Integration algorithm}

The \emph{Integration algorithm} radio buttons select the data transformation
algorithm.

\begin{description}

\item[Madgwick]
We need to write about what this does.
\item[Simple integration]
We need to write about what this does.
\item[Extended Kalman]
We need to write about what this does.

\end{description}





\chapter{Troubleshooting}

This section provides explanation and troubleshooting tips for each error
message produced by \name{}.

\begin{description}

\item[ASA read failed, using 1 adjustment]
\textbf{expand this}
The IMU magnetometers 
This error message is likely an indication that the Arduino code is not
communicating correctly with the magnetometer. It may be necessary to reset the
Arduino.

\item[failed to create connection, aborting]
The program failed to establish a serial connection with the Arduino.  Try
resetting the Arduino.  Make sure that the Arduino is correctly powered and
connected to the PC (Section \ref{sec:wiring}), and that the correct code is
installed on the Arduino (Section \ref{sec:installarduinocode}).

\item[handshake failed]
Even though the PC may have established a valid serial connection to the
Arduino, the data exchange protocol used by \name{} failed to establish a
communication handshake with the Arduino.  Try resetting the Arduino.  Make sure
that the Arduino is correctly powered and connected to the PC (Section
\ref{sec:wiring}), and that the correct code is installed on the Arduino
(Section \ref{sec:installarduinocode}).

\item[invalid csv file]
The program attempted to read a .csv file (Section \ref{sec:fileformats}), but
the format of the data in the file was not valid.

\item[invalid file type:\ldots]
There was an attempt either to save or to load a filetype other than
\csv{} and \hdf{}. As discussed in Section \ref{sec:fileformats},
\csv{} and \hdf{} are the only file formats supported by \name.

\item[no Arduino found]
The program searched for Arduinos on all serial ports but didn't find any.
\name{} uses \texttt{serial.tools.list\_ports} (
\url{http://pyserial.readthedocs.io/en/latest/tools.html}) for this scan.

\item[no IMUs detected, aborting]
The Arduino did not detect any attached IMUs.  After the Arduino is initialized,
it attempts to determine the number of IMUs by sending a WHOAMI request while
signaling each of the three legal chip select lines (Section \ref{sec:wiring}).
It then sends a message to the PC reporting the number of IMUs detected. This
error is reported if the Arduino does not receive a WHOAMI response on any of
the chip select lines.  Make sure that any IMUs are correctly wired to the
Arduino (Section \ref{sec:wiring}) and that the correct code is installed on the
Arduino (Section \ref{sec:installarduinocode}). Try resetting the Arduino.

\item[rx failed, no data read from serial]
The PC tried to receive data from the Arduino but failed. Perhaps no data was
transmitted, or perhaps only data that violates the communication protocol used
by \name{} was received. Try resetting the Arduino and make sure that the
correct code is installed on the Arduino (Section \ref{sec:installarduinocode}).

\item[unknown sample received:\dots]
The PC expected to receive a packet containing a data sample, but the packet
either had the wrong type or the wrong length. If this occurs only briefly at
the beginning of recording, it can probably be ignored. Otherwise, try resetting
the Arduino.

\item[unable to determine number of IMUs, aborting]
The program failed to determine how many IMUs are attached to the Arduino. After
the Arduino is initialized, it attempts to determine the number of IMUs by
sending a WHOAMI request while signaling each of the three legal chip select
lines (Section \ref{sec:wiring}). It then sends a message to the PC reporting the
number of IMUs detected. This error is reported if the PC program sends a
command to the Arduino to initialize, but does not receive a message reporting
the number of IMUs detected. Make sure that any IMUs are correctly wired to the
Arduino (Section \ref{sec:wiring}) and that the correct code is installed on the
Arduino (Section \ref{sec:installarduinocode}). Try resetting the Arduino.



\end{description}




\end{document}

